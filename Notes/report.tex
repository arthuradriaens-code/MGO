\documentclass{article}

\usepackage[english]{babel} 
\usepackage{amsmath,amsfonts,amssymb}
\usepackage{graphicx}
\usepackage{braket}

\title{Notes on MGO}
\author{Arthur Adriaens}

\begin{document}
\maketitle

\section{GO}
GO assumes $\lambda$ is the shortest lenghtscale, suppose a stationary wavefield $\psi$ is governed by a linear wave equation
\begin{equation}
    \hat{\mathcal{D}}(\mathbf{x},-i\partial_\mathbf{x})\psi(\mathbf{x})
    \label{eq:BasicGO}
\end{equation}
With $D$ the dispersion kernel, and suppose further that $\psi$
can be partitioned into a rapidly varying phase $\theta$ and a
slowly varying envelope $\phi$:
\begin{equation}
    \psi(\mathbf{x}) = \phi(\mathbf{x})e^{i\theta(\mathbf{x})}
\end{equation}
Then it can be shown that $\theta$ and $\phi$ asymptotically
satisfy the following two relations:

\begin{enumerate}
    \item The local dispersion relation:
\begin{equation}
    \mathcal{D} [\mathbf{x},\partial_\mathbf{x} \theta(\mathbf{x})] = 0
\end{equation}
    \item The envelope transport equation
\begin{equation}
    2\mathbf{v}(\mathbf{x})^\intercal \partial_\mathbf{x}\phi(\mathbf{x}) + [\nabla\cdot\mathbf{v}(\mathbf{x})]\phi(\mathbf{x}) = 0
\end{equation}
\end{enumerate}

\noindent
Where $\mathcal{D}(\mathbf{x},\mathbf{k})$ is the Weyl symbol of
$\hat{\mathcal{D}}(\mathbf{x},-i\partial_x)$ (appendix \ref{appendix:Weyl}) and 
\begin{equation}
    \mathbf{v}(\mathbf{x}) \stackrel{.}{=} \partial_\mathbf{k}\mathcal{D}(\mathbf{x},\mathbf{k})|_{\mathbf{k}=\partial_\mathbf{x}\theta(\mathbf{x})}
\end{equation}
is proportional to the local group velocity.
If we neglect dissipation, thus assuming
$\hat{D}$ to be Hermitian and consequently, both 
$\mathcal{D}$ and $\mathbf{v}$ real. Then, the envelope transport equation can be cast as a conservation relation:
\begin{equation}
    \nabla\cdot[|\phi(\mathbf{x})|^2\mathbf{v}(\mathbf{x})] = 0
\end{equation}
I.e conservation of wave action flux.

The local dispersion relation defines a (2N-1) dimensional
volume in the 2N-D phase space with coordinates ($\mathbf{x},\mathbf{k}$). For coherent wavefields that have a single wavevector $\mathbf{k}(\mathbf{x})$ (or a finite superposition of such wavevectors), one can identify
\begin{equation}
    \label{eq:theta}
    \mathbf{k}(\mathbf{x}) = \partial_\mathbf{x}\theta(\mathbf{x})
\end{equation}
Such that $\mathbf{k} $ is restricted to an N-D surface
contained within the (2N-1)-D volume (which N-D surface is
dictated by initial conditions). This N-D surface is called
the \textit{ray manifold} which is a Lagrangian manifold.
In particular this means that all vectors $\{\mathbf{T}_j\}$ tangent to it
satisfy
\begin{equation}
    \mathbf{T}_j^\intercal \mathtt{J}_{2N}  \mathbf{T}_{j'} = 0
\end{equation}
Where we have introduced the 2Nx2N matrix
\begin{equation}
    \mathtt{J}_{2N} = 
    \begin{pmatrix}
        0_N & I_N \\
        -I_N & 0_N \\
    \end{pmatrix}
\end{equation}
The ray manifold is a central object in GO and MGO, as such it
will be useful to have an explicit construction of it.
This is provided by the ray (Hamilton's) equations:
\begin{eqnarray}
    \partial_\xi \mathbf{x} &= \partial_\mathbf{k}\mathcal{D}(\mathbf{x},\mathbf{k})\\
    \partial_\xi \mathbf{k} &= -\partial_\mathbf{x}\mathcal{D}(\mathbf{x},\mathbf{k})\\
\end{eqnarray}
The family of solution trajectories [$\mathbf{x}(\xi),\mathbf{k}(\xi)$] for a corresponding family
of initial conditions [$\mathbf{x}(0),\mathbf{k}(0)$] then trace out the ray manifold.
Since the ray manifold is N-D, let's introduce a set of N-D coordinates 
$\mathbf{\tau}$ such that it can be paramterized as [$\mathbf{x}(\mathbf{\tau}),\mathbf{k}(\mathbf{\tau})$].
We shall choose $\tau_1 = \xi$ as a "longitudinal" coordinate along each ray and the remaining
$\mathbf{\tau}_\perp  \stackrel{.}{=} (\tau_2,...,\tau_N)$ as "transverse" coordinates that describe the different
initial conditions of each ray.

Along this family of rays, the envelope transport equation takes the form
\begin{equation}
    \label{eq:re-formed envelope}
    2j(\mathbf{\tau})\partial_{\tau_1}\phi(\mathbf{\tau}) + \phi(\mathbf{\tau})\partial_{\tau_1}j(\mathbf{\tau}) 
\end{equation}
Where we have introduced the Jacobian determinant of the ray trajectories
\begin{equation}
    j(\mathbf{\tau}) \stackrel{.}{=} \text{det  } \partial_{\mathbf{\tau}} \mathbf{x}(\mathbf{\tau})
\end{equation}
Equation \ref{eq:re-formed envelope} can be formally solved to yield
\begin{equation}
    \label{eq:phi}
    \phi(\mathbf{\tau}) = \phi_0(\mathbf{\tau}_\perp)\sqrt{\frac{ j_0(\mathbf{\tau_\perp}) }{ j(\mathbf{\tau}) }}
\end{equation}
This thus states that
\begin{equation}
    |\phi(\mathbf{\tau})|^2|\mathbf{v}|\text{d}A
\end{equation}
is constant along the ray, i.e action is conserved for an infinitesimal "ray
tube".  Having determined $\phi$ from equation \ref{eq:phi} and $\theta$ from
integrating the rays using Hamilton's equations and then using equation
\ref{eq:theta}, the full field $\Psi$ can be constructed by summing over all rays 
that arrive at a given $\mathbf{x}$, i.e
\begin{eqnarray}
    \Psi(\mathbf{x}) &= \sum_{t \in \mathbf{\tau}(\mathbf{x})} \phi(\mathbf{t}) e^{i\theta(\mathbf{t})} \\
                     &\equiv \sum_{t \in \mathbf{\tau}(\mathbf{x})}\phi_0(\mathbf{\tau}_\perp)\sqrt{\frac{ j_0(\mathbf{\tau_\perp}) }{ j(\mathbf{\tau}) }} e^{i\int \mathbf{k}^\intercal \text{d}\mathbf{x}}
\end{eqnarray}
With $\mathbf{\tau}(\mathbf{x})$ the multi-valued formal function inverse of $\mathbf{x}(\mathbf{\tau})$, clearly
the GO field diverges where the jacobian determinant tends to zero,
or equivalently where
\begin{equation}
    \text{det  } \partial_{\mathbf{\mathbf{x}}} \mathbf{k} \equiv \text{det  } \partial_{\mathbf{\mathbf{x}}\mathbf{x}} \mathbf{\theta}  \rightarrow \infty
\end{equation}
such locations are called \textit{caustics}.
\section{Metaplectic Geometrical Optics}
Rather than describing waves as propagating in some
configuration space with coordinates $\mathbf{x}$ according to
pseudo-differential equation of the form (\ref{eq:BasicGO}), it
is more natural to describe waves as state vectors $\ket{\psi}$
in a Hilbert space being acted upon by operators. Then, partial
differential equations that govern wavefields can be understood
as projections of the invariant wave equations 
\begin{equation}
    \hat{D}(\hat{\mathbf{x}},\hat{\mathbf{k}})\ket{\psi} = \ket{0}
\end{equation}
on a particular basis.

\section{Ray Tracing}
The GO equations can be solved by finding phase space trajectories
$\mathbf{z}(\mathbf{\tau}) = (\mathbf{x}(\mathbf{\tau}),\mathbf{k}(\mathbf{\tau}))^\intercal$ satisfying the local dispersion  relation. Such trajectories are called \textit{rays}.
Here $\mathbf{\tau} = (\tau_1,\mathbf{\tau}_\perp)^\intercal$ where $\tau_1$ is the longitudinal time parameter and $\mathbf{\tau}_\perp = (x_2^ {(0)},x_3^ {(0)})^\intercal$ are the perpendicular initial coordinates of the ray.
Given an initial condition $\mathbf{z}(0,\mathbf{\tau}_\perp) = (\mathbf{x}_0,\mathbf{k}_0)^\intercal$, a ray can be found from Hamilton's ray equations:
\begin{eqnarray}
    \partial_{\tau_1}\mathbf{x}(\tau_1) &= -\partial_{\mathbf{k}}\mathcal{D}(\mathbf{x},\mathbf{k})\\
    \partial_{\tau_1}\mathbf{k}(\tau_1) &= -\partial_{\mathbf{x}}\mathcal{D}(\mathbf{x},\mathbf{k})
\end{eqnarray}
We can launch a finite family of rays on a discrete $\mathbf{\tau}_\perp$-grid,
thus populating the space
\appendix
\section{Wigner-Weyl transform}
\label{appendix:Weyl}
The Wigner-Weyl transform (WWT, denoted as $\mathbb{W}$)
maps a given operator $\hat{A}(\hat{\mathbf{z}})$ to a corresponding phase-space function $\mathcal{A}(\mathbf{z})$ , called the 
\textit{Weyl symbol} of $\hat{A}$, as
\begin{eqnarray}
    \mathcal{A}(\mathbf{z}) &= \mathbb{W}[\hat{A}(\hat{\mathbf{z}})]\\
                            &\stackrel{.}{=} \int d\mathcal{\mathbf{\zeta} }\frac{i \mathbf{\zeta}^\intercal \mathtt{J_{2N}\mathbf{z}}}{(2\pi)^N} \text{tr}[e^{-i \mathbf{\zeta}^\intercal \mathtt{J}_{2N} \hat{\mathbf{z}}}\hat{A}]
\end{eqnarray}
With the integral taken over the phase space. and
\begin{equation}
    \mathtt{J}_{2N} = 
    \left(
    \begin{matrix}
        0_N & I_N\\
        -I_N & 0_N\\
    \end{matrix}
    \right)
\end{equation}
The inverse WWT maps a phase-space function to an operator:
\begin{eqnarray}
    \hat{A}(\mathbf{z}) &= \mathbb{W}^{-1}[\mathcal{A}(\mathbf{z})]\\
                        &\stackrel{.}{=} \int \frac{d\mathbf{z}'d\mathcal{\mathbf{\zeta} }}{(2\pi)^{2N}}\mathcal{A}(\mathbf{z}')e^{-i \mathbf{\zeta}^\intercal \mathtt{J}_{2N} \mathbf{z}' + i \mathbf{\zeta}^\intercal \mathtt{J}_{2N} \hat{\mathbf{z}}}
\end{eqnarray}

\end{document}
